\documentclass[twocolumn]{ctexbook}

\usepackage[nodeprecatedcode]{dnd}

\usepackage[singlelinecheck=false]{caption}
\usepackage{lipsum}
\usepackage{multicol}
\usepackage{listings}
\usepackage{shortvrb}
\usepackage{stfloats}
\usepackage{zhnumber}

\captionsetup[table]{labelformat=empty,font={sf,sc,bf,},skip=0pt}

\MakeShortVerb{|}
\newfontfamily\crn{Courier New}
\lstset{%
  basicstyle=\crn,
  language=[LaTeX]{TeX},
  breaklines=true,
}

\title{The Dark \LaTeX{} \\
\large An Example of the dndbook Class}
\author{The rpgTeX Team}
\date{2020/04/21}

\begin{document}

\frontmatter

\maketitle

\tableofcontents

\mainmatter

\part{布局}

\chapter{章节}

\DndDropCapLine{本}{宏包是用来帮助您}为世界上最伟大的角色扮演游戏的第五版撰写排版漂亮的文章的。
它首先把\LaTeX{}中默认的小节排版变为读者更加熟悉的形式。
章节排版则如上所示。

\section{小节}
使用小节(section)来把章节分成大量相关的文字。

\subsection{小小节}
使用小小节(subsection)来更进一步地为读者拆分信息。

\subsubsection{小小小节}
小小小节(subsubsection)依然具有块级的标头,比它更细的划分中的标头是内联的。

\paragraph{段落}
在核心规则书中很少见到段落(paragraph)格式,但是如果您更加喜欢段落式的排版,也可以使用它。

\subparagraph{子段落}
带有段落缩进的子段落(subparagraph)格式对读者而言可能更加熟悉。

\section{特殊的小节}

本模块也包括了用来辅助多行小节标头排版的函数:|\DndFeatHeader|用于专长(Feat)的排版、|\DndItemHeader|用于魔法物品与陷阱的排版、|\DndSpellHeader|用于法术的排版。

\DndFeatHeader{排版智者 Typesetting Savant}[先决条件:\LaTeX{}分发版]

你获得了帮助你最喜欢的游戏之一的原始材料进行排版的宏包。
让你获得以下增益:

\begin{itemize}
	\item 你为排版新内容所做的智力检定具有优势;
	\item 当你未能通过为排版新内容所做的智力检定时,你可以在该宏包的网站上寻求帮助。
\end{itemize}

\DndItemHeader{福之翎笔 Foo's Quill}{奇物,珍稀}

该翎笔具有三发充能。
持有它时,你可以使用一次动作消耗一发充能。
该翎笔会跃出你的手中,并根据你所处的情况撰写一份契约。

该翎笔每日黎明时回复1d3发已消耗的充能。

\DndSpellHeader%
  {精致排版 Beautiful Typesetting}
  {四环,幻术}
  {1动作}
  {5尺}
  {S,M (墨水与羊皮纸)}
  {直至被解除}

你能够角任意长度的书面消息变形为一张美丽的卷轴。
任何能够看到该卷轴的生物必须进行一次感知豁免检定,若失败则被你魅惑。

被你魅惑的生物的视线不能离开该卷轴,且不会自愿地远离它。
目标在其回合结束时可以进行一次感知豁免检定,若成功则不再被魅惑。

\section{地图区域}
地图区域函数|\DndArea|和|\DndSubArea|会自动为区域标号。

\DndArea{霍姆莱特村 Village of Hommlet}
名为霍姆莱特的村庄。
(这是第一版附带的新手模组的发生地。)

\DndSubArea{迎客少女酒馆 Inn of the Welcome Wench}
迎客少女酒馆位于霍姆莱特村中。

\DndSubArea{铁匠铺 Blacksmith's Forge}
村庄中还有一个铁匠铺。

\DndArea{福之城堡 Foo's Castle}
这是福的住所,是一间由泥土和木棍搭成的破旧小屋。

\DndSubArea{护城壕}
壕沟上横跨着一块木板。

\DndSubArea{入口}
一个五尺高的洞,借助屋顶的洞透过的光线,从其中可以看见泥土地面。

\chapter{文本框}

这个模块含有三个分离文本用来吸引读者注意的环境。
|DndReadAloud|用来标记游戏主持人应当大声朗读出的文本。

\begin{DndReadAloud}
  当你靠近这个模块时,能感受到无数代人为它而付出的血与泪。
  当你输入第一个字时,一种温暖的感觉随之而来。
\end{DndReadAloud}

\section{侧栏}
其余两个环境为|DndComment|和|DndSidebar|。
|DndComment|可以断开,能够在文本中内联使用。

\begin{DndComment}{这是一个评论框!}
  |DndComment|是为文本提供最简单高亮的文本框。
  它缺少|DndSidebar|周围的装饰,但是能够在两栏之间断开。
\end{DndComment}

|DndSidebar|不能断开,因此最好使用浮动(float)来固定在页面一角。

\begin{DndSidebar}[float=!b]{看,DndSidebar!}
  |DndSidebar|用在侧边栏中。
  它不会在两栏之间断开,最好和figure环境一切使用来浮动在页面的一角,这样文本可以环绕它。
\end{DndSidebar}

\section{表格}
|DndTable|为偶数行上色,且默认定宽为行宽。

\begin{DndTable}[header=好看的表格]{XX}
    表头  & 表头 \\
    值  & 值 \\
    值  & 值 \\
    值  & 值
\end{DndTable}

\chapter{怪物与NPC}

% Monster stat block
\begin{DndMonster}[float*=b,width=\textwidth + 8pt]{福之怪 Monster Foo}
  \begin{multicols}{2}
    \DndMonsterType{中型异怪(占位变量),中立邪恶}

    % If you want to use commas in the key values, enclose the values in braces.
    \DndMonsterBasics[
        armor-class = {9(带\emph{法师护甲}12)},
        hit-points  = {\DndDice{3d8 + 3}},
        speed       = {30尺,飞行30尺},
      ]

    \DndMonsterAbilityScores[
        str = 12,
        dex = 8,
        con = 13,
        int = 10,
        wis = 14,
        cha = 15,
      ]

    \DndMonsterDetails[
        %saving-throws = {Str +0, Dex +0, Con +0, Int +0, Wis +0, Cha +0},
        %skills = {Acrobatics +0, Animal Handling +0, Arcana +0, Athletics +0, Deception +0, History +0, Insight +0, Intimidation +0, Investigation +0, Medicine +0, Nature +0, Perception +0, Performance +0, Persuasion +0, Religion +0, Sleight of Hand +0, Stealth +0, Survival +0},
        %damage-vulnerabilities = {cold},
        %damage-resistances = {bludgeoning, piercing, and slashing from nonmagical attacks},
        %damage-immunities = {poison},
        %condition-immunities = {poisoned},
        senses = {黑暗视觉60尺,被动察觉10},
        languages = {通用语、地精语、地下通用语},
        challenge = 1,
      ]
    % Traits
    \DndMonsterAction{天生施法}
    施法关键属性为魅力(法术豁免DC 12,法术攻击命中+4)。它天生可施展以下法术而不需要任何相应的材料成分:
    \begin{DndMonsterSpells}
      \DndInnateSpellLevel{迷踪步}
      \DndInnateSpellLevel[3]{云雾术,魔绳术}
      \DndInnateSpellLevel[1]{鉴定术}
    \end{DndMonsterSpells}

    \DndMonsterAction{施法}
    其为二级施法者。施法关键属性为魅力(法术豁免DC 12,法术攻击命中+4)。它准备了以下法术:
    \begin{DndMonsterSpells}
      \DndMonsterSpellLevel{剑刃防护,火焰箭,光亮术,电爪}
      \DndMonsterSpellLevel[1][3]{燃烧之手,法师护甲,护盾术}
    \end{DndMonsterSpells}

    \DndMonsterSection{行动}
    \DndMonsterAction{多重攻击}
    进行两次近战攻击。

    %Default values are shown commented out
    \DndMonsterAttack[
      name=匕首,
      %distance=both, % valid options are in the set {both,melee,ranged},
      %type=weapon, %valid options are in the set {weapon,spell}
      mod=+3,
      %reach=5,
      %range=20/60,
      %targets=one target,
      dmg=\DndDice{1d4+1},
      dmg-type=穿刺,
      %plus-dmg=,
      %plus-dmg-type=,
      %or-dmg=,
      %or-dmg-when=,
      %extra=,
    ]

    %\DndMonsterMelee calls \DndMonsterAttack with the melee option
    \DndMonsterMelee[
      name=火舌长剑,
      mod=+3,
      %reach=5,
      %targets=one target,
      dmg=\DndDice{1d8+1},
      dmg-type=挥砍,
      plus-dmg=\DndDice{2d6},
      plus-dmg-type=火焰,
      or-dmg=\DndDice{1d10+1},
      or-dmg-when=若双手共持,
      %extra=,
    ]

    %\DndMonsterRanged calls \DndMonsterAttack with the ranged option
    \DndMonsterRanged[
      name=刺客轻弩,
      mod=+1,
      range=80/320,
      dmg=\DndDice{1d8},
      dmg-type=穿刺,
      %plus-dmg=,
      %plus-dmg-type=,
      %or-dmg=,
      %or-dmg-when=,
      extra={,目标须进行一次DC 15的体质豁免检定,失败则受24 (7d6)毒素伤害,成功伤害减半}
    ]

    % Legendary Actions
    \DndMonsterSection{传奇动作}
    其拥有3 传奇动作用于选择执行下列动作选项。每次执行传奇动作时,每个选项只可以选用一次,且只能在另一生物的回合结束时执行。在它的回合开始时恢复其所有已消耗的传奇动作数。

    \begin{DndMonsterLegendaryActions}
      \DndMonsterLegendaryAction{移动}{最快以其最大速度移动。}
      \DndMonsterLegendaryAction{匕首攻击}{进行一次匕首攻击。}
      \DndMonsterLegendaryAction{创建契约(花费3行动)}{Foo用其所知的语言创造一份契约并向10尺内任意一个生物展示。该生物必须进行一次DC 10的智力豁免检定。失败时,该生物陷入失能直至Foo的下一个回合。不能能理解该契约的语言者的检定具有优势。}
    \end{DndMonsterLegendaryActions}
  \end{multicols}
\end{DndMonster}

使用|DndMonster|环境来为怪物和非玩家角色的资料板进行排版。
此模块提供许多函数来为资料板的内容进行排版。

\part{自定义}

\chapter{颜色}

\begin{table*}[b]
  \caption{\DndFontTableTitle{}本宏包支持的颜色}\label{tab:colors}

  \begin{tabularx}{\linewidth}{lX}
    \textbf{颜色}                  & \textbf{描述} \\
    \rowcolor{PhbLightGreen}
    |PhbLightGreen|                 & 玩家手册第一部分使用的淡绿色(默认) \\
    \rowcolor{PhbLightCyan}
    |PhbLightCyan|                  & 玩家手册第二部分使用的淡青色 \\
    \rowcolor{PhbMauve}
    |PhbMauve|                      & 玩家手册第三部分使用的淡紫色 \\
    \rowcolor{PhbTan}
    |PhbTan|                        & 玩家手册附录使用的浅棕色 \\
    \rowcolor{DmgLavender}
    |DmgLavender|                   & 城主手册第一部分使用的淡紫色 \\
    \rowcolor{DmgCoral}
    |DmgCoral|                      & 城主手册第二部分使用的橘粉色 \\
    \rowcolor{DmgSlateGray}
    |DmgSlateGray| (|DmgSlateGrey|) & 城主手册第三部分使用的蓝灰色 \\
    \rowcolor{DmgLilac}
    |DmgLilac|                      & 城主手册附录使用的紫灰色 \\
    \rowcolor{BrGreen}
    |BrGreen|                       & 基础规则中表格使用的灰绿色 \\
  \end{tabularx}
\end{table*}

该宏包提供几种全局的颜色变量来为|DndComment|、|DndReadAloud|、|DndSidebar|和|DndTable|环境指定样式。

\begin{DndTable}[header=文本框颜色]{lX}
  颜色            &  描述 \\
  |commentcolor|   & |DndComment|背景色 \\
  |readaloudcolor| & |DndReadAloud|背景色 \\
  |sidebarcolor|   & |DndSidebar|背景色 \\
  |tablecolor|     & |DndTable|偶数行的背景色 \\
\end{DndTable}

这些环境也接受一个可选的颜色参数用来为单个实例指定颜色。
可用的核心规则书颜色参见表格\ref{tab:colors}。

\begin{lstlisting}
\begin{DndTable}[color=PhbLightCyan]{cX}
  d8 & Item \\
  1  & Small wooden button \\
  2  & Red feather \\
  3  & Human tooth \\
  4  & Vial of green liquid \\
  5  & Loaded dice \\
  6  & Tasty biscuit \\
  7  & Broken axe handle \\
  8  & Tarnished silver locket \\
\end{DndTable}
\end{lstlisting}

\begin{DndTable}[color=PhbLightCyan]{cX}
  d8 & Item \\
  1  & Small wooden button \\
  2  & Red feather \\
  3  & Human tooth \\
  4  & Vial of green liquid \\
  5  & Loaded dice \\
  6  & Tasty biscuit \\
  7  & Broken axe handle \\
  8  & Tarnished silver locket \\
\end{DndTable}

\section{主题颜色}
使用|\DndSetThemeColor[<color>]|来为|commentcolor|、|readaloudcolor|、|sidebarcolor|和|tablecolor|设置指定的颜色。
不带参数地调用|\DndSetThemeColor|能将以上颜色设置为当前的|themecolor|。
以下例子使用分组将颜色的改变限制在几个文本框中。
分组之后,颜色就回到了开始前的状态。

\begin{lstlisting}
\begingroup
\DndSetThemeColor[PhbMauve]

\begin{DndComment}{This Comment Is in Mauve}
  This comment is in the the new color.
\end{DndComment}

\begin{DndSidebar}{This Sidebar Is Also Mauve}
  The sidebar is also using the new theme color.
\end{DndSidebar}
\endgroup
\end{lstlisting}

\begingroup
\DndSetThemeColor[PhbMauve]

\begin{DndComment}{淡紫色评论}
  这个评论使用新的主题色。
\end{DndComment}

\begin{DndSidebar}{淡紫色侧栏}
  这个侧栏也使用新的主题色。
\end{DndSidebar}
\endgroup

\end{document}
